\section{Linear Combinations, Span, and Basis Vectors}\label{sec:span}
\subsection{An Alternate Way to Think About Vectors}
\begin{itemize}
   \item think about the pair of numbers that describe a vector as scalars individually
   \item each vector is a linear combination of basis vectors
\end{itemize}

\subsection{Choosing Different Basis Vectors}
\begin{itemize}
    \item what other two dimensional vectors can you reach by combining a scaled version of two vectors?
    \item imagine these two new vectors as the "new" basis vectors
\end{itemize}

\subsection{Linear Combinations}
\begin{itemize}
    \item any time two (or more) vectors are scaled and added, it's called a \textbf{linear combination} 
    \item \textit{linear}: if you scale your vectors by all real numbers, the result is an infinite line through the origin and the vector's point
    \item a linear combination is a method of combining the lines created by vectors
\end{itemize}

\subsection{Span}
\begin{itemize}
    \item \textbf{Span}: the set of all possible vectors you can reach with a linear combination of two vectors is their span
    \item what points can we reach using just vector addition and scalar multiplication?
\end{itemize}

\subsection{Vectors vs. Points}
\begin{itemize}
    \item representing vectors as lines can make the visualization crowded, so we can also represent the vector's tip as a point instead
\end{itemize}

\subsection{Span in 3D}
\begin{itemize}
    \item if you have two vectors in 3D, this allows you to reach only a plan as a linear combination of these points
    \item if you add a third vector, you have two possible outcomes
    \begin{itemize}
        \item if the vector is on the already-existing plan, the span remains the same. you can't explore any new area
        \item if your vector is not on the plane, you can reach the entire 3D space
    \end{itemize}
    \item \textbf{linearly dependent}: a vector can be removed without reducing the span; in other words, it can be created by a linear combination of your other vectors
    \item \textbf{linearly independent}: the vector adds another dimension to the span;  it cannot be created by a linear combination of other vectors
\end{itemize}
