\section{Linear Transformations and Matrices}
\subsection{Transformations are Functions}
\begin{itemize}
    \item \textbf{Transformation}: something that takes in an input and produces an output. In the context of linear algebra, take in a vector and produce another vector.
    \item Can think about a transformation as moving every possible input vector to its corresponding output
\end{itemize}

\subsection{What makes a transformation "linear"?}
\begin{itemize}
    \item a transformation is \textbf{linear} is all lines remain lines (they are not curved) and the origin remains fixed
    \item grid lines remain parallel and evenly spaced
\end{itemize}

\subsection{Matrices}
\begin{itemize}
    \item we can think about all linear transformations in terms of transformations on the basis vectors
    \item Example 1:
    \begin{align}
        \mathbf{v} &= -1\mathbf{i} + 2\mathbf{j} \\
        L(\mathbf{v}) &= -1 \begin{bmatrix}
             1 \\
            -2 
            \end{bmatrix} + 2 \begin{bmatrix}
            3 \\
            0 \\
            \end{bmatrix} \\
        L(\mathbf{v}) &= \begin{bmatrix}
            -1(1) + 2(3) \\
            -1(-2) + 2(0)
        \end{bmatrix} \\
        L(\mathbf{v}) &= \begin{bmatrix}
        5 \\
        2 
        \end{bmatrix}
    \end{align}
    \item Example 2:
    \begin{align}
        i = \begin{bmatrix}
        -1 \\
        1
        \end{bmatrix} \text{and } j = \begin{bmatrix}
        -2 \\ 
        -1
        \end{bmatrix} \\
        \text{Input vector } = \begin{bmatrix}
        -3 \\ 
        -1
        \end{bmatrix} \\
        \mathbf{i} = -3\mathbf{i} - 1\mathbf{j} \\
        L(\mathbf{v}) = -3 [-1, 1]^T + -1 [-2, -1]^T \\
        L(\mathbf{v}) = \begin{bmatrix}
        -3(-1) + -1(-2) \\
        -3(1) + -1(-1)
        \end{bmatrix} \\
        L(\mathbf{v}) = \begin{bmatrix}
            5 \\
            -2
        \end{bmatrix}
    \end{align}
    \item as long as we know how the transformation acts on $i$ and $j$, we can do this with any vector
    \item we can generalize with $x$ and $y$ as our vector now, and $a, b, c, d$ as our transformation vector components:
    \begin{align}
        L(\mathbf{v}) &= x \begin{bmatrix}
            a \\
            c \\
        \end{bmatrix} + y \begin{bmatrix}
            b \\
            d \\
        \end{bmatrix} \\
        L(\mathbf{v}) &= \begin{bmatrix}
            ax + by \\
            cx + dy \\
        \end{bmatrix}
    \end{align}
    \item so, we can write a two-dimensional linear transformation with just four numbers: the two coordinates where $i$ lands and the two coordinates where $j$ lands 
    \item we can then package these four numbers into a matrix; column 1 is where $i$ lands, and column 2 is where $j$ lands
    \item we can rewrite our previous generalization as "matrix-vector multiplication":
    $$ \begin{bmatrix}
        a & b \\
        c & d
    \end{bmatrix}
    \begin{bmatrix}
        x \\
        y
    \end{bmatrix} =
    x \begin{bmatrix}
        a \\
        c
    \end{bmatrix}
    + y \begin{bmatrix}
        b \\
        d
    \end{bmatrix} =
    \begin{bmatrix}
        ax + by \\
        cx + dy \\
    \end{bmatrix} $$
\end{itemize}

\subsection{Linearly Dependent Columns}
\begin{itemize}
    \item if the vectors that $i$ and $j$ land on are linearly dependent, the transformation squishes all of 2 space onto the line where they sit (one-dimensional span) 
\end{itemize}

\subsection{Formal Properties}
\begin{itemize}
    \item \textbf{Additivity Property}: $L(\mathbf{v} + \mathbf{w}) = L(\mathbf{v}) + L(\mathbf{w})$ - summing two vectors, \textit{then} transforming is equivalent to transforming the vectors and then summing
    \item \textbf{Scaling Property}: $L(s \mathbf{v}) = s L(\mathbf{v})$ - transforming a scaled vector is equivalent to transforming a vector and then scaling
\end{itemize}