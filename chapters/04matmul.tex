\section{Matrix Multiplication}
\subsection{Recap}
\begin{itemize}
    \item linear transformations: functions with vectors as inputs and outputs
    \item transform space in a way that keeps gridlines parallel, evenly spaced
    \item columns of a matrix represent how each basis vector is transformed
\end{itemize}

\subsection{Composition of Transformations}

\begin{itemize}
    \item Composition: applying one transformation after another
    \item can compute a composition with two matrix-vector multiplies
    $$ AB = C \rightarrow A(Bv) = C(v) $$
    \item \textbf{Multiplying two matrices has the geometric meaning of applying one transformation after another.}
    \begin{example}
        Composition of a rotation and a shear:
        $$ f(x) = \text{shear} \\
        g(x) = \text{rotate} \\
        f(g(x)) = \text{shear}(\text{rotate}(x)) $$
        $$ \begin{bmatrix}
        1 & 1 \\
        0 & 1 
        \end{bmatrix}
        \begin{bmatrix}
        0 & -1 \\
        1 & 0 
        \end{bmatrix}
        = \begin{bmatrix}
        1 & -1 \\
        1 & 0
        \end{bmatrix} $$
    \end{example}
    \begin{example}
        Multiply two matrices, $M_1$ and $M_2$. First, examine how the $i$ vector is transformed:
        $$ M_2(M_1(i)) = \begin{bmatrix}
        0 & 2 \\
        1 & 0 
        \end{bmatrix}
        \begin{bmatrix}
        1 & -2 \\
        1 & 0 
        \end{bmatrix}
        \begin{bmatrix}
        1 \\
        0 
        \end{bmatrix} \newline
        = 1 \begin{bmatrix}
        0 \\
        1 
        \end{bmatrix} +
        1 \begin{bmatrix}
        2 \\
        0
        \end{bmatrix} =
        \begin{bmatrix}
        2 \\
        1
        \end{bmatrix} $$
        Apply the same process to $j$ to see where it lands:
        $$ M_2(M_1(j)) = \begin{bmatrix}
        0 & 2 \\
        1 & 0
        \end{bmatrix}
        \begin{bmatrix}
        1 & -2 \\
        1 & 0 
        \end{bmatrix}
        \begin{bmatrix}
        0 \\
        1
        \end{bmatrix} \\
        = -2 \begin{bmatrix}
        0 \\
        1
        \end{bmatrix}
        + 0 \begin{bmatrix}
            2 \\
            0
        \end{bmatrix} \\
        = \begin{bmatrix}
        0 \\
        -2
        \end{bmatrix} $$
        We can represent the matrix-vector product of the $i$ and $j$ vectors as a matrix:
        $$ M_2(M_1) = \begin{bmatrix}
        2 & 0 \\
        1 & -2
        \end{bmatrix}$$ 
    \end{example}
\end{itemize}

\subsection{General Form}
\begin{itemize}
    \item Using the above example, we express a matrix multiplication in general form:
    \begin{example}
        We can express a matrix multiplication in general form:
        \begin{align}
            M_2(M_1) &= \begin{bmatrix}
                a & b \\
                c & d
            \end{bmatrix}
            \begin{bmatrix}
                e & f \\
                g & h 
            \end{bmatrix} \\
            &= \begin{bmatrix}
                ae + bg & af + bh \\
                ce + dg & cf + dh
            \end{bmatrix}
        \end{align}
    The resulting matrix is equivalent to vector products $M_2 M_{1_{:, 1}}$ and $M_2 M_{1_{:, 2}}$, where $M_{1_{:, 1}}$ and $M_{1_{:, 2}}$ are the first and second columns of $M_1$.
    \end{example}
    \item In short, matrix multiplication \textit{is} multiplying one transformation after another.
\end{itemize}

\subsection{Noncommutativity}
\begin{itemize}
    \item The order of transformations changes how the final space looks
    \item Therefore, matrix multiplication is not commutative: $AB \neq BA$
\end{itemize}

\subsection{Associativity}
\begin{itemize}
    \item $(AB)C = A(BC)$
\end{itemize}