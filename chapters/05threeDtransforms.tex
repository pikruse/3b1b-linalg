\section{Three-Dimensional Linear Transformations}
Principles in 2 dimensions extend to 3+ dimensions
\begin{example}
    A linear transformation in 3D:
    $$ \begin{bmatrix}
    2 \\
    6 \\
    -1
    \end{bmatrix} \rightarrow
    L(v) \rightarrow
    \begin{bmatrix}
    3 \\
    2 \\
    0
    \end{bmatrix}$$.
    The transformation \textit{as a whole} is:
    $$ \begin{bmatrix}
        a \\
        b \\
        c
    \end{bmatrix} \rightarrow
    L(v) \rightarrow
    \begin{bmatrix}
        x \\
        y \\
        z
    \end{bmatrix}$$.
\end{example}
\begin{itemize}
    \item We can think of each linear transformation in terms of how it changes the basis vectors.
    \item In a $3 \times 3$ matrix, each column represents instructions for how to scale each basis vector.
\end{itemize}
\begin{example}
    What is the transformation that rotates space 90 degrees counterclockwise around the z-axis?

    Since each column of our matrix represents how the basis vectors are transformed, we can specify what the $x, y$, and $z$ coordinates will look like. $x and y$ will be rotated, while $z$ will be unchanged:
    $$ x' = \begin{bmatrix}
    0 \\
    -1 \\
    0 
    \end{bmatrix};
    y' = \begin{bmatrix}
    1 \\
    0 \\
    0
    \end{bmatrix}; 
    z' = z \rightarrow T = \begin{bmatrix}
    0 & 1 & 0 \\
    -1 & 0 & 0 \\
    0 & 0 & 1
    \end{bmatrix}$$
\end{example}

\subsection{Combining Transformations}
The logic for two dimensions applies once again: two $3 \times 3$ matrices being multiplied is equivalent to transforming the basis vectors first by the
leftmost matrix, and next by the right matrix.